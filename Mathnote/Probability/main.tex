\documentclass{article}

\usepackage{amsmath, amsthm, amssymb, amsfonts}
\usepackage{thmtools}
\usepackage{graphicx}
\usepackage{setspace}
\usepackage{geometry}
\usepackage{float}
\usepackage{hyperref}
\usepackage[utf8]{inputenc}
\usepackage[english]{babel}
\usepackage{framed}
\usepackage[dvipsnames]{xcolor}
\usepackage{tcolorbox}

\colorlet{LightGray}{White!90!Periwinkle}
\colorlet{LightOrange}{Orange!15}
\colorlet{LightGreen}{Green!15}
\colorlet{LightBlue}{Blue!15}
\colorlet{LightRed}{Red!15}



\newcommand{\HRule}[1]{\rule{\linewidth}{#1}}
\newcommand{\N}{\mathbb{N}}
\newcommand{\R}{\mathbb{R}}
\newcommand{\Prob}{\mathbb{P}}

\declaretheoremstyle[name=Theorem,]{thmsty}
\declaretheorem[style=thmsty,numberwithin=section]{theorem}
\tcolorboxenvironment{theorem}{colback=LightGray}

\declaretheoremstyle[name=Proposition,]{prosty}
\declaretheorem[style=prosty,numberlike=theorem]{proposition}
\tcolorboxenvironment{proposition}{colback=LightOrange}

\declaretheoremstyle[name=Definition,]{definsty}
\declaretheorem[style=definsty,numberlike=theorem]{definition}
\tcolorboxenvironment{definition}{colback=LightGreen}

\declaretheoremstyle[name=example,]{examsty}
\declaretheorem[style=examsty,numberlike=theorem]{example}
\tcolorboxenvironment{example}{colback=LightRed}

\declaretheoremstyle[name=remark,]{Remarksty}
\declaretheorem[style=Remarksty,numberlike=theorem]{remark}
\tcolorboxenvironment{remark}{colback=LightBlue}

\setstretch{1.2}
\geometry{
    textheight=9in,
    textwidth=5.5in,
    top=1in,
    headheight=12pt,
    headsep=25pt,
    footskip=30pt
}

% ------------------------------------------------------------------------------

\begin{document}

% ------------------------------------------------------------------------------
% Cover Page and ToC
% ------------------------------------------------------------------------------

\title{ \normalsize \textsc{}
		\\ [2.0cm]
		\HRule{1.5pt} \\
		\LARGE \textbf{\uppercase{Probability theory}
		\HRule{2.0pt} \\ [0.6cm] \LARGE{Study Note} \vspace*{10\baselineskip}}
		}
\date{}
\author{\textbf{Author} \\ 
		Sam Ren \\
		Grinnell College \\
		\today }

\maketitle
\newpage

\tableofcontents
\newpage

% ------------------------------------------------------------------------------

\section*{Acknowledgements}
I finsihed this study notes by simply collected the concepts and terms from @brightsideofmaths youtube videos, here is the link: \hyperlink{https://www.youtube.com/playlist?list=PLBh2i93oe2qswFOC98oSFc37-0f4S3D4z}{Video link}. At the same time I would like to sicerely thank to my teachers and friends. Without them this note would not be done.
\newpage

\section{Measure Theory}
\textbf{This section is not necessary for later studying but it helps with understanding those abstract concepts}








\newpage
\section{Probability Measure}
	Probability Measure is the measure with total mass 1, where we call it sample space $\Omega,Area(\Omega)=1$. In math we write:
	\begin{equation*}
		\mathbb{P}: A\to \R 
	\end{equation*}
	Where $A$ is the collection of subsets, then we have:
	\begin{equation*}
		\Prob (\Omega)=1,
		\Prob(A)\in [0,1]
	\end{equation*}
It is easy to notice that if two sets($A,B$) are disjointed($A\cap B=\emptyset$) then we have:
\begin{equation*}
	\Prob( A\cup B)=\Prob(A)+\Prob(B)
\end{equation*}

\begin{theorem}
	If we have a sequence of disjoin sets $A_i,i\in \N$ then we have:
	\begin{equation*}
		\Prob(\bigcup_{i=1}^{\infty}A_i)=\sum_{i=1}^{\infty}\Prob(A_i)
	\end{equation*}
\end{theorem}

\begin{definition}
	Let $\Omega$ be a set. A collection of subsets $\mathcal{A}\subseteq \mathcal{P}(\Omega)$ ( $\mathcal{P}$ is the power set) is called a sigma algebra($\sigma-$Algebra) if:
	\begin{enumerate}
		\item $\Omega,\emptyset\in \mathcal{A}$
		\item If $A\in \mathcal{A}$ then $A^c\in \mathcal{A}$
		\item If $A_1,A_2,A_3,...\in \mathcal{A}$ then $\bigcup_{i=1}^{\infty}A_i\in \mathcal{A}$
	\end{enumerate}
\end{definition}

In Probability theory we call the elements in $\sigma$-algebra events. Then we are going to define the probability measure properly.
\begin{definition}
	Let $\mathcal{A}\subseteq \mathcal{P}(\Omega)$ is a sigma-algebra. A map $\Prob: \mathcal
	{A}\to [0,1]$ is called probability measure if:
	\begin{enumerate}
		\item $\Prob(\Omega)=1$
		\item $\Prob(\emptyset)=0$
		\item $\Prob(\bigcup_{i=1}^{\infty}A_i)=\sum_{i=1}^{\infty}\Prob(A_i)$ where $i\neq j$
	\end{enumerate} 
\end{definition}

\begin{example}
	If we have $\Omega=\{1,2,3,4,5,6\}$, then $\mathcal{A}=\mathcal{P}(\Omega)$ we have:
	\begin{equation*}
		\Prob: \mathcal{A}\to [0,1],\Prob(A):=\frac{|A|}{|\Omega |}
	\end{equation*}
\end{example}

\begin{proof}
	Prove that $\Prob( A^c)=1-\Prob(A)$
\end{proof}

\subsection{Discrete \& Continuous Probability measure}
We are going to discuss two situations where the sample space is either discrete or continuous. Where at here the term \textbf{Discrete} means the sample space is either countable or finite. For term \textbf{Continuous} the sample space is either infinite or uncountable.


\begin{table}[hbt]
\centering
  \begin{tabular}{l|c|c}
   Type & discrete & continuous \\
    \hline
 Sample space   & $\Omega$ countable or finite& $\Omega\subseteq \R^n$ uncountable, $\Omega\in \mathcal{B}(\R^n)$ \\
 $\sigma-$algebra   & $\mathcal{A}=\mathcal{P}(\Omega)$ & $\mathcal{A}=\mathcal{B}(\Omega)$\\
 Prob Measure & $\Prob: \mathcal{A}\to [0,1]$ & $\Prob: \mathcal{A}\to [0,1]$\\
 Determined by & $\Prob(\{\omega\}),\forall \omega\in \Omega$ & \\
mass function & $(p_\omega)_{\omega\in \Omega}$ with $p_\omega\geq 0,\sum_{\omega\in \Omega}p_\omega=1$ & $f:\Omega\to \R$ with $f(x)\geq 0,\int_{ \Omega}f(x)dx=1$ \\
 Definition & $\Prob(A)= \sum_{\omega\in A}p_\omega=1$ & $\Prob(A)=\int_Af(x)dx$
  \end{tabular}
  \caption{Discrete and continuous sample space explaination table}
\end{table}
For continous probability measure we need to ensure that $\Omega$ is measuralbe which we are going to discuss later. Here is an example for continuous measure:

\begin{example}
	For $\Omega=[0,2]$ we have prob density function $f(x)=\frac{1}{2},f:\Omega\to \R$. Hence:
	\begin{equation*}
		\Prob(\Omega)=\int_0^2f(x)dx=\frac{1}{2} \int_0^21dx=1
	\end{equation*} 
	Therefore we would have conclusion:
	\begin{equation*}
		\Prob(A)=\int_A f(x)dx=\frac{1}{2}\int_A1dx=\frac{1}{2}\text{(Lebesgue measure)Length of A}
	\end{equation*}
	We would discuss the Lebesgue measure later.
\end{example}

\subsection{Binomial Distribution}
A binomial(two outcomes) distribution is foundamental to the probability theory, and it also require a definition(For now this definition is improper):
\begin{definition}
	A binomial distribution can be determined by following features:
	\begin{enumerate}
		\item There is no order
		\item With replacements (That is, does not change the sample space)
		\item There are only two outcomes
	\end{enumerate}
	If it is a binomial distribution we denote it as $\Prob=B(n,p)$. For a binomial distribution we have:
	\begin{equation*}
		\Omega=\{0,1,2,3,...,n\},\Prob(\{k\})= {n\choose k}p^k(1-p)^{n-k}
	\end{equation*}
\end{definition}

\subsection{Product Probability Spaces}
In previous sections we already introduced the probability measure($\Prob$), the $\sigma$-algebra($\mathcal{A}$) and sample space($\Omega$). In this section we are going to introduce the probability space which is important in precisely proof and mathematics content.
\begin{definition}
	 Probability space is a triple combination$(\Omega_n,\mathcal{A}_n,\Prob_n), n\in \{1,2,3,...\}$. Then the product space($\Omega,\mathcal{A},\Prob$) is given by:\begin{itemize}
	 	\item $\Omega=\Omega_1\times\Omega_2\times\Omega_3\times...=\prod_{i\in \N}\Omega_i$
	 	\item $\mathcal{A}$=$\sigma(\Omega_1\times{A}_2\times\Omega_3\times...)$ The choice of $A_n\in \mathcal{A}$ is not restricted by 2, it can be any number. Hence we just simply replace one $\Omega_i$ with $A_i$ to construct.
	 	\item $\Prob(A_1\times A_2\times {A}_3\times ...\times A_m\times \Omega_{m+1}\times \Omega_{m+2}\times ...)=\Prob(A_1)\cdot\Prob(A_2)\cdot...\cdot \Prob(A_m)$
	 \end{itemize}

\end{definition}
You can find examples in this \hyperlink{https://www.youtube.com/watch?v=7WvdpFqptZk&list=PLBh2i93oe2qswFOC98oSFc37-0f4S3D4z&index=5}{Link}. 

\subsection{Hypergeometric Distribution(Multivariant)}
For this distribution we consider the sample size $n$ and without replacement and unordered. Let's begin with an example:
\begin{example}
Let elements in set $S$ be the color of balls:
	\begin{equation*}
	S=\{0,1,2,3\}
\end{equation*}
If we have two 0 and 1 for others then we have a $(2,1,1,1)$. At the same time there are many functions where mapping set $S$ to $\N$. We denote the set of these functions as $\N^S$. We now can define the sample space:
\begin{equation*}
	\Omega=\{(k_s)_{s\in S}\in \N^S| \sum_{s\in S}k_s=n        \}
\end{equation*}
Let's denote $N_s$ as the number of balls in color $s$ and $N$ as the total number:
Then we would have:
\begin{equation*}
	\Prob((k_s)_{s\in S})=\frac{\prod_{s\in S}{N_s\choose k_s}}{{N\choose k}}
\end{equation*}

\end{example}

\begin{definition}
	For hypergeometric distribution, we have (variant)set $S=\{0,1,2,...,n\}$ and sample space $\Omega=\{0,1,2,3,...,n\}$:
	\begin{equation*}
		\Prob((k_s)_{s\in S})=\frac{\prod_{s\in S}{N_s\choose k_s}}{{N\choose k}}
	\end{equation*}
\end{definition}





\section{Conditional Probability}







\newpage

% ------------------------------------------------------------------------------
% Reference and Cited Works
% ------------------------------------------------------------------------------

\bibliographystyle{IEEEtran}
\bibliography{References.bib}

% ------------------------------------------------------------------------------

















\end{document}
